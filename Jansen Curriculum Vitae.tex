\documentclass[12pt,a4paper]{article}

\usepackage[utf8]{inputenc}
\usepackage{amsmath}
\usepackage{amsfonts}
\usepackage{amssymb}
\usepackage{libertine}
\usepackage{tabto}
\usepackage{xcolor}
\usepackage[colorlinks]{hyperref}
	\hypersetup{
    colorlinks,
    linkcolor={red!50!black},
    citecolor={blue!60!black},
    urlcolor={blue!40!black}
	}
\usepackage[super]{nth}
\usepackage{titlesec}
	\titleformat{\section}{\normalfont\Large\itshape}{\thesection}{0em}{}
	\titleformat{\subsection}{\normalfont\normalsize\bfseries}{\thesubsection}{1em}{}
	%\titlespacing*{\subsection}{0pt}{10pt}{5pt}
\usepackage{enumitem}
	\newlist{collections}{description}{1}
	\setlist[collections]{
		font=\bfseries\normalcolor\ttfamily, 
		before*={\normalfont},  
		topsep=0pt,
		partopsep=0pt,
		parsep=0pt,
		itemsep=0pt
		}
\usepackage{savetrees}
\usepackage{courier}

\begin{document}

\begin{center}
	{\LARGE M. Andrew Jansen}\\
	\medskip
	{\Large Curriculum Vit\ae}\\
	Updated:~\textit{\today}
\end{center}

\section*{Personal Information}

	\begin{description}
		\item [Position] \tabto*{2cm} Ph.D. Candidate
		\item [Address] \tabto*{2cm} School of Life Sciences, PO Box 874501, Arizona State University, Tempe, AZ 85287-4501, USA
		\item [Phone] \tabto*{2cm} (386) 631-1305
		\item [E-mail] \tabto*{2cm} \href{mailto:majanse1@asu.edu}{majanse1@asu.edu}
	\end{description}

\section*{Research Interests}

	Insect evolution, biomechanics, and systematics, with emphasis on weevils (Coleoptera: Curculionoidea: Curculionid\ae);
	mathematical and finite element modeling of insect cuticle;
	structural adaptation and morphological optimization;
	mechanical behavior of biomaterials;
	biomimetic design of robotic systems and materials.

\section*{Education}

	\begin{description}
		\item [2014 - 2017] Ph.D. Candidate, Arizona State University, Evolutionary Biology.
		\item [2012 - 2014] M.Sc., Arizona State University, Biology.
		\item [2007 - 2011] B.S., University of Florida, Entomology and Nematology.
	\end{description}

\section*{Peer Reviewed Publications}

	\begin{description}
		\itemsep0.5em
		\item \textbf{Jansen, M.A.}, Luck, K., Campbell, J., Amor, H.B., \& D. Aukes. 2017. Bio-inspired robot design considering load-bearing and kinematic ontogeny of Chelonioidea sea turtles. In \textit{Biomimetic and Biohybrid Systems}. \href{http://www.springer.com/us/book/9783319635361}{p. 216-229}
		\item Luck, K., \textbf{Jansen, M.A.}, Campbell, J., Aukes, D., \& H.B. Amor. 2017. From the lab to the desert: fast prototyping and learning of robot locomotion. \textit{Proceedings of Robotics: Science and Systems}. \href{http://www.roboticsproceedings.org/rss13/p75.html}{13: p. 75-83}.
		\item \textbf{Jansen, M.A.}, Singh, S.S., Chawla, N., \& N.M. Franz. 2016. A multilayer micromechanical model of the cuticle of \textit{Curculio longinasus} Chittenden, 1927 (Coleoptera: Curculionid\ae). \textit{Journal of Structural Biology}. \href{http://www.sciencedirect.com/science/article/pii/S1047847716300922}{195: p. 139-158.}
		\item Singh, S.S., \textbf{Jansen, M.A.}, Franz, N.M., \& N. Chawla. 2016. Microstructure and nanoindentation of the rostrum of \textit{Curculio longinasus} Chittenden, 1927 (Coleoptera: Curculionid\ae). \textit{Materials Characterization}. \href{http://www.sciencedirect.com/science/article/pii/S1044580316301619}{118: p. 206-211.}
		\item \textbf{Jansen, M.A.} \& S.E. Halbert. 2016. Key to Florida Alydid{\ae} (Hemiptera: Heteroptera) and selected exotic pest species. \textit{Insecta Mundi}. \href{http://journals.fcla.edu/mundi/article/view/87952/84644}{0476: p. 1-14.}
		\item \textbf{Jansen, M.A.} \& N.M. Franz. 2015. Phylogenetic revision of \textit{Minyomerus} Horn, 1876 sec. Jansen \& Franz, 2015 (Coleoptera, Curculionid\ae) using taxonomic concept annotations and alignments. \textit{ZooKeys}. \href{http://zookeys.pensoft.net/articles.php?id=6001}{528: p. 1-133.}
	\end{description}

\section*{Manuscripts in Preparation}

	\begin{description}
		\itemsep0.5em
		\item \textbf{Jansen, M.A.} \& N.M. Franz. Descriptions of four new species of \textit{Minyomerus} Horn, 1876 (Coleoptera: Curculionid\ae), with notes on their distribution and phylogeny. \textit{In preparation}.
	\end{description}

\section*{Products Developed}

	\subsection*{C-Turtle}
		\begin{description}
			\itemsep0em
			\item[Website] \href{https://sites.google.com/view/c-turtle/}{https://sites.google.com/view/c-turtle/}
			\item[Design] \href{https://drive.google.com/file/d/0BxntR7XVPIVqekR3Sjcyd1hkUm8/view}{Version 1.0 Cut-files}
			\item[License] \href{https://creativecommons.org/licenses/by/4.0/}{Attribution 4.0 International} (CC BY 4.0)
		\end{description}

	\subsection*{Patent Applications}

		\begin{description}
			\itemsep0.5em
			\item Aukes, D., Amor, H.B., Luck, K., \textbf{Jansen, M.A.}, \& J. Campbell, \textit{inventors}; Arizona State University, Skysong Innovations, \textit{assignee}. 2017. United States provisional patent application for systems and methods for rapid-prototyped robotic devices. \textit{US Patent Application No. 62/597,276}. Filed 11 December.
		\end{description}

	\subsection*{Featured Media}

		\begin{description}
			\itemsep0.25em
			\item Adams, D. 2017. ``An army of these odd-looking robotic 'turtles' might help rid the world of landmines''. \textit{Digital Trends}. \href{https://www.digitaltrends.com/cool-tech/robot-turtles-detect-landmines/}{26 May}.
			\item Ander, J. 2017. ``Landmine-clearing Pi-powered C-Turtle''. \textit{Raspberry Pi Official Blog}. \href{https://www.raspberrypi.org/blog/landmine-c-turtle/}{26 July}.
			\item Coledewey, D. 2017. ``These flat-pack turtlebots will crawl across minefields for safety's sake''. \textit{Tech Crunch}. \href{https://techcrunch.com/2017/05/25/these-flat-pack-turtlebots-will-crawl-across-minefields-for-safetys-sake/}{25 May}.
			\item Crookes, D. 2017. ``C-TURTLE''. \textit{The MagPi Magazine}: Issue 63 \href{https://www.raspberrypi.org/magpi/c-turtle/}{1 November}.
			\item DeLisle, J.J. 2017. ``Raspberry-Pi-powered turtle robot learns to navigate new terrains on its own - From planetary exploration to swarm robotic landmine sensing, C-Turtle's possibilities are endless''. \textit{Electronic Products}. \href{https://www.electronicproducts.com/Robotics/AI/Raspberry_Pi_powered_turtle_robot_learns_to_navigate_new_terrains_on_its_own.aspx}{11 August}.
			\item Fagan, K. 2017. ``The landmine-detecting robot `turtle'". \textit{BBC News}. \href{http://www.bbc.com/news/av/technology-40296297/the-soft-3d-printed-robot-that-could-come-to-the-rescue}{22 July}.
			\item Horsey, J. 2017. ``Raspberry Pi used to create C-Turtle, landmine clearing robot''. \textit{Geeky Gadgets}. \href{https://www.geeky-gadgets.com/landmine-clearing-robot-27-07-2017/}{27 July}.
			\item Kety, S. 2017. ```C-Turtle', the 3D printed robot whose movements are similar to a sea turtle''. \textit{3D Adept News}. \href{https://3dadept.com/c-turtle-the-3d-printed-robot-whose-movements-are-similar-to-a-sea-turtle/}{16 August}.
			\item Koslow, T. 2017. ``Out of the shell - C-Turtle: the paper turtle robot that can detect landmines''. \textit{All3DP}. \href{https://all3dp.com/landmine-detecting-robot-c-turtle/}{20 August}.
			\item Lavars. N. 2017. ``Turtle-bot teaches itself to waddle through the desert''. \textit{New Atlas}. \href{https://newatlas.com/turtle-robot-self-learning-waddle/49738/}{26 May}.
			\item Ludacer, R. 2017. ``Researchers are using robotic sea turtles to find land mines''. \textit{Tech Insider}. \href{https://www.facebook.com/techinsider/videos/787578874773804/}{10 June}.
			\item Ray, A. 2017. ``A new turtle explorer - This \$70 robot that mimics a sea-turtle may eventually reach Mars''. \textit{Quartz}. \href{https://qz.com/1053078/this-70-robot-that-mimics-a-sea-turtle-may-eventually-reach-mars/}{15 August}.
			\item Massaouden, L. 2017. ``C-Turtle, le robot tortue en carton qui doit un jour explorer Mars''. \textit{Mashable avec France 24}. \href{http://mashable.france24.com/videos/20170825-c-turtle-robot-tortue-carton-exploration-mars}{25 August}.
			\item Mathews, L. 2017. ``Robotic Turtles With Raspberry Pi Brains Are Sniffing Out Land Mines''. \textit{Geek.com}. \href{https://www.geek.com/tech/robotic-turtles-with-raspberry-pi-brains-are-sniffing-out-land-mines-1709339/}{27 July}.
			\item Sabin, D. 2017. ``This crawling C-Turtle robot could hunt for landmines''. \textit{Inverse}. \href{https://www.inverse.com/article/32219-cturtle-robot-sea-turtle-mines}{26 May}.
			\item Reynolds, M. 2017. ``Robotic turtles can be used to detect landmines in the desert''. \textit{New Scientist Magazine}: Issue 3127. \href{https://www.newscientist.com/article/mg23431274-200-robotic-turtles-can-be-used-to-detect-landmines-in-the-desert/}{24 May}.
			\item Sant, J.V. 2017. ``ASU Robotics turns to nature for inspiration''. KPHO Broadcasting Corporation: \textit{3TV/CBS5}. \href{http://www.azfamily.com/story/35595946/asu-robotics-turns-to-nature-for-inspiration}{5 June}.
			\item Scott, C. 2017. ``Partially 3D printed C-Turtle robots crawl and adapt in the desert''. \textit{3Dprint.com}. \href{https://3dprint.com/184523/3d-printed-c-turtle-robots/}{17 August}.
			\item Seckel, S. 2017. ``Technology comes from collaboration between computer science, mechanical engineering and biology''. \textit{ASU Now}. \href{https://asunow.asu.edu/20170525-solutions-asu-designed-c-turtle-robot-teaches-itself-get-around}{25 May}.
			\item Seckel, S. 2017. ``ASU-designed C-Turtle robot teaches itself to get around''. \textit{ASU Now}. \href{https://asunow.asu.edu/20170525-solutions-asu-designed-c-turtle-robot-teaches-itself-get-around}{25 May}.
			\item Wehner, M. 2017. ``These robotic turtles could save your life''. \textit{New York Post}. \href{https://nypost.com/2017/05/25/these-robotic-turtles-could-save-your-life/}{25 May}.
			\item Unknown - `Hackster Staff'. 2017. ``Nature-inspired C-Turtle robot waddles the desert with ease''. \textit{Hackster}. \href{https://blog.hackster.io/nature-inspired-c-turtle-robot-waddles-the-desert-with-ease-3061cbc19b36}{26 May}.
			\item Unknown - `Gadget Junkie'. 2017. ``C-Turtle: cardboard turtle robot with Raspberry Pi''. \textit{gadgetify}. \href{http://www.gadgetify.com/c-turtle-cardboard-turtle-robot/}{27 July}.
			\item Unknown - `Robot Man'. 2017. ``C-Turtle cardboard robot turtle learns to navigate different terrains''. \textit{Robotic Gizmos}. \href{http://www.roboticgizmos.com/c-turtle-robot-turtle/}{27 July}.
		\end{description}

\section*{Conference Presentations}

	\begin{description}
		\itemsep0.5em
		\item \textbf{Jansen, M.A.}, Chawla, N., \& N.M. Franz. 2017. ``Fracture mechanics and evolution of resilient cuticle in the rostrum of \textit{Curculio} Linnaeus, 1758 ''. Annual Meeting of the Entomological Society of America, Denver, CO.
		\item \textbf{Jansen, M.A.} \& N.M. Franz. 2017. ``Evolutionary mechanics of the rostrum in \textit{Curculio} Linnaeus, 1758''. Annual Meeting of the Willi Hennig Society, St. Petersburg, FL.
		\item \textbf{Jansen, M.A.} \& N.M. Franz. 2016. ``Why the long face? Insights into the mechanical behavior of the rostrum in the genus \textit{Curculio} Linnaeus, 1758''. International Congress of Entomology, Orlando, FL.
		\item \textbf{Jansen, M.A.}, Singh, S.S., Chawla, N., \& N.M. Franz. 2015. ``Mechanical Behavior of the Rostrum of \textit{Curculio} Linnaeus, 1758 (Coleoptera: Curculionid\ae)''. Annual Meeting of the Entomological Society of America, Minneapolis, MN.
		\item \textbf{Jansen, M.A.} \& N.M. Franz. 2014. ``A phylogenetic revision of \textit{Minyomerus} Horn, 1876, and \textit{Piscatopus} Sleeper, 1960 (Coleoptera: Curculionid\ae: Entimin\ae: Tanymecini)''. Annual Meeting of the Entomological Society of America Pacific Branch, Tucson, AZ.
		\item \textbf{Jansen, M.A.} \& N.M. Franz. 2013. ``A phylogenetic revision of \textit{Minyomerus} Horn, 1876, and \textit{Piscatopus} Sleeper, 1960 (Coleoptera: Curculionid\ae: Entimin\ae: Tanymecini)''. Annual Meeting of the Entomological Society of America, Austin, TX.
		\item \textbf{Jansen, M.A.} \& N.M. Franz. 2013. ``A phylogenetic revision of \textit{Minyomerus} Horn, 1876, and \textit{Piscatopus} Sleeper, 1960 (Coleoptera: Curculionid\ae)''. 12th Biennial Conference of Science and Management on the Colorado Plateau, Flagstaff, AZ.
	\end{description}

\section*{Society Memberships}

	\begin{description}
		\item [2013 - 2018] Entomological Society of America, Pacific Branch
		\item [2013 - 2018] Coleopterists Society
	\end{description}

\section*{Awards Received}

	\begin{description}
		\item [2017] \$400.00 - ASU School of Life Sciences Fall Travel Award
		\item [2017] \$195.00 - ASU Q2 Graduate College Travel Award
		\item [2017] \$6,000.00 - ASU Evolutionary Biology Doctoral Program Summer Fellowship
		\item [2016] \$400.00 - ASU School of Life Sciences Fall Travel Award
	\end{description}

\section*{Academic Service}

	\subsection*{Manuscript Reviewer}
	
		\begin{description}
			\item Coleopterists Society Monographs (Patricia Vaurie Series) 
			\item The Pan-Pacific Entomologist 
			\item Zootaxa
		\end{description}

	\subsection*{Book Chapter Reviews}

		\begin{description}
			\item ``Weevils (Coleoptera: Dryophthorid\ae, Brachycerid\ae, Erirhinid\ae, Curculionid\ae) of the Prairie Ecozone in Canada''.
					Robert S. Anderson, Patrice Bouchard, \& Hume Douglas.
					In Volume 4 of \textit{Arthropods of Canadian Grasslands}.
		\end{description}

	\subsection*{Community Outreach} 

		\begin{description}
			\item [2013 - 2016] ASU - SoLS Night of the Open Door
			\item [2013 - 2016] ASU - IAFSE Engineering Open House
			\item [2014 - 2015] ASU - SoLS Graduate Partners in Science Education
		\end{description}
		
	\subsection*{Insect Identification Services}
	
		\begin{description}
			\item [2017 - 2018] US Department of Agriculture - Tempe, AZ, USA
			\item [2017] Greater Good, Madrean Discovery Expedition - Caj\'{o}n Bonito, SO, M\'{e}xico
			\item [2014] Madrean Discovery Expedition - Patagonia, AZ, USA
			\item [2013] Madrean Discovery Expedition - Sierra la P\'{u}rica, SO, M\'{e}xico
			\item [2013] US National Park Service, BioBlitz - Joshua Tree National Park, CA, USA
			\item [2012] Madrean Discovery Expedition - Sierra Aconchi, SO, M\'{e}xico
		\end{description}

\section*{Teaching Appointments}

	\begin{tabbing}
	\hspace{2cm}\=\hspace{6.5cm}\=\kill
	\textbf{Course} \> \textbf{Subject} \> \textbf{Semester} \\
	BIO 201 \> Human Anatomy and Physiology \> Spring - 2017 \\ 
	BIO 281 \> Biology (\nth{1}~Semester) \> Fall - 2016 \\ 
	BIO 182 \> Biology (\nth{2}~Semester) \> Summer - 2016 \\ 
	BIO 181 \> Biology (\nth{1}~Semester) \> Spring - 2016 \\ 
	BIO 386 \> Entomology \> Fall - 2015, 2014, 2013 \\ 
	BIO 282 \> Biology (\nth{2}~Semester for Majors) \> Spring 2014
	\end{tabbing} 

\section*{Field and Museum Work}

	\subsection*{Field Work}
		\begin{description}
			\item [United States] \tabto*{3cm} AZ, CA, CO, FL, GA, NM, NV, SC, TX, UT (2010-2017)
			\item [Mexico] \tabto*{3cm} SO (2012, 2013, 2017)
			\item [Guatemala] \tabto*{3cm} AV, BV, CM, CQ, ES, GU, HU, IZ, JA, PR, QC, QZ, SA, SO, SR, SU, TO, ZA (2014)
		\end{description}

	\subsection*{Collections Visited}
		\begin{collections}
			\item [ASUT] USA, Arizona, Tempe, Arizona State University, Hasbrouck Insect Collection
			\item [BYUC] USA, Utah, Provo, Brigham Young University, Monte L. Bean Life Science Museum
			\item [CASC] USA, California, San Francisco, California Academy of Sciences
			\item [CSCA] USA, California, Sacramento, California State Collection of Arthropods
			\item [CSDS] USA, California, Baker, Desert Studies Center
			\item [CSUC] USA, Colorado, Fort Collins, Colorado State University
			\item [CWOB] USA, Arizona, Green Valley, Charles W. O'Brien Collection
			\item [EMEC] USA, California, Berkeley, University of California, Essig Museum of Entomology
			\item [FSCA] USA, Florida, Gainesville, Division of Plant Industry, Florida State Collection of Arthropods
			\item [FSMC] USA, Florida, Gainesville, University of Florida, Florida Museum of Natural History
			\item [LBOB] USA, Arizona, Green Valley, Lois B. O'Brien Collection
			\item [MGCL] USA, Florida, Gainesville, University of Florida, McGuire Center for Lepidoptera and Biodiversity
			\item [NAUF] USA, Arizona, Flagstaff, Northern Arizona University
			\item [NMSU] USA, New Mexico, Las Cruces, New Mexico State University, Museum of Southwestern Biology
			\item [NVDA] USA, Nevada, Reno, Nevada State Department of Agriculture
			\item [RLAC] USA, California, El Dorado Hills, Rolf L. Aalbu Collection
			\item [SWRS] USA, Arizona, Portal, Southwestern Research Station
			\item [TAMU] USA, Texas, College Station, Texas Agricultural and Mechanical University
			\item [TTUZ] USA, Texas, Lubbock, Texas Tech University
			\item [UAIC] USA, Arizona, Tucson, University of Arizona
			\item [UCDC] USA, California, Davis, University of California, R.M. Bohart Museum of Entomology
			\item [UCRC] USA, California, Riverside, University of California, Entomology Research Museum
			\item [UNMC] USA, New Mexico, Albuquerque, University of New Mexico
			\item [UMNH] USA, Utah, Salt Lake City, University of Utah, Utah Museum of Natural History
			\item [UVGC] Guatemala, Guatemala City, Universidad del Valle de Guatemala, Colleci\'{o}n de Artr\'{o}podos
		\end{collections}

\section*{Employment History}

	\begin{description}
		\item [2012] Museum Technician - Florida State Collection of Arthropods, University of Florida
		\item [2012] Museum Technician - McGuire Center for Lepidoptera, University of Florida
		\item [2011] Research Technician - Honeybee Research and Extension Laboratory, University of Florida
		\item [2011] Research Assistant - Division of Insect Behavior, USDA-ARS, Gainesville, FL
		\item [2009 - 2011] Senior Counsellor - Center for Precollegiate Education and Training, University of Florida
	\end{description}

\end{document}